\section{Haldy}

V nasledujúcom texte sa budeme zaoberať rôznymi druhmi prioritných front. Popíšeme \emph{$d$-árnu haldu} ako 
základnú modifikáciu binárnej haldy, \emph{ľavicovú haldu} a niektoré druhy samoupravujúcich sa háld, konkrétne 
\emph{skew haldu} a \emph{párovaciu haldu}.
Halda je vo všeobecnosti \emph{zakorenený strom} s vrcholmi obsahujúcimi kľúče reprezentujúce dáta. Dôležitá je 
zakladná podmienka haldy, ak vrchol $p(x)$ je otcom vrcholu $x$, potom
$\hbox{\emph{kľúč}}(p(x)) \leq \hbox{\emph{kľúč}}(x)$\footnote{Bez ujmy na všeobecnosti budeme uvažovať o \emph{min haldách},
teda v koreni sa bude nachádzať najmenší prvok. Podobnými úvahami by sme text mohli rozšíriť o \emph{max haldy}
s najväčším prvkom v koreni.}.

Štandardné operácie, ktoré haldy podporujú, a ktorými sa budeme zaoberať pri každej dátovej štuktúre, sú:
\begin{itemize}
%\item $\mathop{\mathit{createHeap}}$ -- vytvorí prázdnu haldu;
\item $\ins(x)$ -- vloží vrchol s kľúčom $x$;
\item $\mathop{\mathit{findMin}}$ -- vráti minimum, t.j.~hodnotu kľúča v koreni;
\item $\mathop{\mathit{deleteMin}}$ -- odstráni vrchol s najmenším kľúčom, t.j.~koreň;
\item $\dec(v, \Delta)$ -- zníži kľúč vrcholu $v$ o $\Delta\geq0$;
\end{itemize}

Niektoré haldy navyše implementujú operáciu $\meld(i, j)$, ktorá spojí haldu $i$ s haldou $j$.

\input haldy/daryheap.tex
\input haldy/leftist.tex
\input haldy/skewheap.tex
\input haldy/pairheap.tex
