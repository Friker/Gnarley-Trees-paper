\section{História}
Každá vizualizovaná operácia (insert/delete/...) na dátovej štruktúre pozostáva
z niekoľkých krokov. Jednou z noviniek v projekte je možnosť vrátiť sa pri
prehliadani operácií o niekoľko krokov späť (história krokov), resp. vrátiť späť
celé operácie (história operácií).

Niekedy sa stáva, že nedočkavý užívateľ rýchlo prekliká cez celú vizualizáciu
operácie a pritom si nestihne uvedomiť, aké zmeny sa vykonali na danej dátovej
štruktúre. Inokedy je operácia taká rozsiahla, že niektoré dôležité zmeny si
nevšimne. Vtedy by bolo užitočné pozrieť si vizualizáciu ešte raz (alebo
niekoľkokrát). Tento problém rieši história krokov. Užívateľ má možnosť vrátiť
sa späť o jeden krok (tlačidlo "Späť"/"Previous") alebo preskočiť na ľubovoľný
krok po kliknutí na zodpovedajúci komentár.

História krokov a operácií je atomická. Krok/operácia sa vykoná/vráti celý(-á)
alebo vôbec, pričom stav dátovej štruktúry korešponduje s pozíciou v histórii.
To umožňuje po vrátení celej operácie vykonať inú operáciu. Táto vlastnosť je
užitočná najmä v prípade vykonania operácie (prípadne zmazania celej dátovej
štruktúry) omylom.

