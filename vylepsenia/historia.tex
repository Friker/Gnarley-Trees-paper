\subsection{História krokov}
Každá vizualizovaná operácia na dátovej štruktúre pozostáva
z niekoľkých krokov. Jednou z noviniek v projekte je možnosť vrátiť sa pri
prehliadaní operácií o niekoľko krokov späť, resp.\ vrátiť späť
celé operácie.

Niekedy sa stáva, že nedočkavý používateľ rýchlo prekliká cez celú vizualizáciu
a pritom si nestihne uvedomiť, aké zmeny sa vykonali na danej dátovej
štruktúre. Inokedy je operácia taká rozsiahla, že niektoré dôležité zmeny si
nevšimne. Vtedy by bolo užitočné pozrieť si vizualizáciu ešte raz (alebo
niekoľkokrát). Tento problém rieši história krokov. Používateľ má možnosť vrátiť
sa späť o jeden krok (tlačidlo "Späť"/"Previous") alebo preskočiť na ľubovoľný
krok po kliknutí na zodpovedajúci komentár.

História krokov a operácií je atomická. Krok/operácia sa vykoná/vráti celý(-á)
alebo vôbec, pričom stav dátovej štruktúry korešponduje s pozíciou v histórii.
To umožňuje po vrátení celej operácie vykonať inú operáciu. Táto vlastnosť je
užitočná najmä v prípade vykonania operácie (prípadne zmazania celej dátovej
štruktúry) omylom.

