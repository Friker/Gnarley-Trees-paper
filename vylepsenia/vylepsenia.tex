\section{Rozšírenie predošlej práce}
%jednotlive vizualizacie a implementovane ficurie
%co sa zmenilo od bakalarky? zoomovanie, komentare, tree layouty, historia a nove ds

toto treba kompletne vymyslieť a napísať a potom tam dakde pichnúť:

\paragraph{Tesnejšie vykresľovanie grafov.}
V pôvodnej verzii programu sa stromy vykresľovali tak, že vertikálna súradnica 
predstavovala hĺbku v strome a horizontálna poradie vrcholu v \emph{
inorderovom prechode} stromu. Toto je ale spôsob, ktorý 
nešetrí priestor a pri štruktúrach ako písmenkový strom by výsledné stromy 
vyzerali škaredo. Preto sme sa rozhodli pre stromy implementovať 
alternatívny spôsob rozloženia, ktorý vymyslel \citet{reingold} a pre $n$-árne 
stromy rozšíril \citet{walker}. Tieto rozloženia vykresľujú vrcholy stromov čo 
najtesnejšie, pričom dodržujú tieto pravidlá: 
\begin{itemize} 
\item vrcholy v rovnakej hĺbke sú vykreslené na jednej priamke a priamky 
určujúce jednotlivé úrovne sú rovnobežné; 
\item poradie synov je zachované; 
\item otec leží v strede nad najľavejším a najpravejším synom; 
\item izomorfné podstromy sa vykreslia identicky až na presunutie;
\item ak vo všetkých vrcholoch vymeníme poradie všetkých synov, výsledný strom 
sa vykreslí zrkadlovo.
\end{itemize}
