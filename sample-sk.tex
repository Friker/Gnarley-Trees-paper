% !Mode:: "TeX::UTF-8"
\documentclass[slovensky]{svk}
%% ak pisete po slovensky, pouzijete namiesto horneho riadku
%% \documentclass[slovensky]{svk}

\begin{document}
\title{Príprava článku na ŠVK}

\author{Daniel Ševčovič\inst{1}
\email{sevcovic@fmph.uniba.sk}
\and 
Tomáš Vinař\inst{2}
\email{vinar@fmph.uniba.sk}}
%% vsimnite si, ze u autorov sa nepisu tituly
%% prikaz \inst sluzi ako odkaz do zoznamu institucii
%% (vid. nizsie)

%% skolitela nepiste medzi autorov, ale v tejto casti
%% ak praca nema skolitela, jednoducho vynechajte
\supervisor{Tom\'a\v{s} Plachetka\inst{3}
\email{plachetk@dcs.fmph.uniba.sk}}

%% nasleduje kratka verzia nazvu clanku a 
%% zoznam autorov (bez krstnych mien)
%% tieto informacie sa zobrazuju v hlavicke
\titlerunning{Príprava článku na ŠVK}
\authorrunning{Ševčovič and Vinař}

\institute{
Katedra aplikovanej matematiky a štatistiky,
FMFI UK,
Mlynská Dolina,
842~48~Bratislava,
\and
Katedra aplikovanej informatiky,
FMFI UK,
Mlynská Dolina
842~48~Bratislava
\and
Katedra informatiky,
FMFI UK,
Mlynská Dolina
842~48~Bratislava}

\maketitle

\begin{abstract}
V tomto článku prezentujeme jednoduchý dokument, ktorý
možno použiť ako príklad ako pripraviť článok na ŠVK
v systéme \LaTeX.

\keywords{\LaTeX makrá, sadzba textu, ŠVK}
\end{abstract}

\section{Úvod a príklady}
Tento článok je príkladom, ako použiť makrá ŠVK v \LaTeX{}u.
Článok poukazuje na preferencie ŠVK čo sa týka štýlu a formátovania
rôznych častí textu, ako napríklad citácií, matematických vzorcov
a pod. 

\subsection{Sample text}
Let $S=[s_{ij}]$ ($1\leq i,j\leq n$) be a $(0,1,-1)$-matrix
of order $n$. Then $S$ is a {\em sign-nonsingular matrix}
(SNS-matrix) provided that each real matrix with the same
sign pattern as $S$ is nonsingular. There has been
considerable recent interest in constructing and
characterizing SNS-matrices \cite{bs}, \cite{klm}. There
has also been interest in strong forms of
sign-nonsingularity \cite{djd}. In this paper we give a new
generalization of SNS-matrices and investigate some of
their basic properties.
 
Let $S=[s_{ij}]$ be a $(0,1,-1)$-matrix of order $n$ and
let $C=[c_{ij}]$ be a real matrix of order $n$. The pair
$(S,C)$ is called a {\em matrix pair of order} $n$.
Throughout, $X=[x_{ij}]$ denotes a matrix of order $n$
whose entries are algebraically independent indeterminates
over the real field. Let $S\circ X$ denote the Hadamard
product (entrywise product) of $S$ and $X$. We say that the
pair $(S,C)$ is a {\em sign-nonsingular matrix pair of
order} $n$, abbreviated SNS-{\em matrix pair of order} $n$,
provided that the matrix \[A=S\circ X+C\] is nonsingular
for all positive real values of the $x_{ij}$.  If $C=O$
then the pair $(S,O)$ is a SNS-matrix pair if and only if
$S$ is a SNS-matrix.  If $S=O$ then the pair $(O,C)$ is a
SNS-matrix pair if and only if $C$ is nonsingular. Thus
SNS-matrix pairs include both nonsingular matrices and
sign-nonsingular matrices as special cases.
 
\subsection{An enumeration list}
In this paper we consider the evaluation of integrals of the 
following forms:
\begin{equation}
\int_a^b \left( \sum_i E_i B_{i,k,x}(t) \right)
         \left( \sum_j F_j B_{j,l,y}(t) \right) dt,\label{problem}
\end{equation}
\begin{equation}
\int_a^b f(t) \left( \sum_i E_i B_{i,k,x}(t) \right) dt,\label{problem2}
\end{equation}
where $B_{i,k,x}$ is the $i$th B-spline of order $k$ defined over the
knots $x_i, x_{i+1}, \ldots, x_{i+k}$.
We will consider B-splines normalized so that their integral is one.
The splines may be of different orders and
defined on different knot sequences $x$ and $y$.
Often the limits of integration will be the entire real line, $-\infty$
to $+\infty$. Note that (\ref{problem}) is a special case of (\ref{problem2})
where $f(t)$ is a spline.


There are five different methods for calculating (\ref{problem})
that will be considered:
\begin{enumerate}
\item Use Gauss quadrature on each interval.
\item Convert the integral to a linear combination of
      integrals of products of B-splines and provide a recurrence for
      integrating the product of a pair of B-splines.
\item Convert the sums of B-splines to piecewise
      B\'{e}zier format and integrate segment
      by segment using the properties of the Bernstein polynomials.
\item Express the product of a pair of B-splines as a linear combination
      of B-splines.
      Use this to reformulate the integrand as a linear combination
      of B-splines, and integrate term by term.
\item Integrate by parts.
\end{enumerate}
Of these five, only methods 1 and 5 are suitable for calculating 
(\ref{problem2}). The first four methods will be touched on and the 
last will be discussed at length.


\subsection{Some displayed equations and \{{\tt eqnarray}\}s}
     By introducing the product topology on  $R^{m \times m} \times
R^{n \times n}$  with the induced inner product
\begin{equation}
\langle (A_{1},B_{1}), (A_{2},B_{2})\rangle := \langle A_{1},A_{2}\rangle 
+ \langle B_{1},B_{2}\rangle,\label{eq2.10}
\end{equation}
we calculate the Fr\'{e}chet derivative of  $F$  as follows:
\begin{eqnarray}
 F'(U,V)(H,K) &=& \langle R(U,V),H\Sigma V^{T} + U\Sigma K^{T}\nonumber\\
             && - P(H\Sigma V^{T} + U\Sigma K^{T})\rangle \nonumber \\
         &=& \langle R(U,V),H\Sigma V^{T} + U\Sigma K^{T}\rangle\nonumber \\
&=& \langle R(U,V)V\Sigma^{T},H\rangle + \nonumber\\
  &&    \langle \Sigma^{T}U^{T}R(U,V),K^{T}\rangle.    \label{eq2.11}
\end{eqnarray}

In the middle line of (\ref{eq2.11}) we have used the fact that the range of
$R$ is always perpendicular to the range of $P$.  The gradient $\nabla F$  of
$F$, therefore,  may be interpreted as the
pair of matrices:
\begin{eqnarray}
 \nabla F(U,V) &=& (R(U,V)V\Sigma^{T},R(U,V)^{T}U\Sigma )\nonumber\\
 && \in R^{m \times m} \times R^{n \times n}.   \label{eq2.12}
\end{eqnarray}

Thus, the vector field
\begin{equation}
\frac{d(U,V)}{dt} = -g(U,V) 	\label{eq2.15}
\end{equation}
defines a steepest descent flow on the manifold  ${\cal O} (m) \times
{\cal O} (n)$ for the objective function  $F(U,V)$.

 
\section{Main results} 
 
Let $(S,C)$ be a matrix pair of order $n$.  The determinant
\[\det (S\circ X+C)\] 
is a polynomial in the indeterminates of $X$ of degree at
most $n$ over the real field. We call this polynomial the
{\em indicator polynomial} of the matrix pair $(S,C)$
because of the following proposition.
 
\begin{theorem} 
\label{th:prop} 
The matrix pair $(S,C)$ is a {\rm SNS}-matrix pair if and
only if all the nonzero coefficients in its indicator
polynomial have the same sign and there is at least one
nonzero coefficient.
\end{theorem} 
 
\begin{proof}
Assume that $(S,C)$ is a SNS-matrix pair.  Clearly the
indicator polynomial has a nonzero coefficient.  Consider a
monomial
\begin{equation} 
\label{eq:mono} 
b_{i_{1},\ldots,i_{k};j_{1},\ldots,j_{k}}x_{i_{1}j_{1}}\cdots
x_{i_{k}j_{k}}
\end{equation} 
occurring in the indicator polynomial with a nonzero
coefficient.  By taking the $x_{ij}$ that occur in
(\ref{eq:mono}) large and all others small, we see that any
monomial that occurs in the indicator polynomial with a
nonzero coefficient can be made to dominate all others.
Hence all the nonzero coefficients have the same sign. The
converse is im-\linebreak mediate. \qquad\end{proof} 
 

For SNS-matrix pairs $(S,C)$ with $C=O$ the indicator
polynomial is a homogeneous polynomial of degree $n$. In
this case Theorem \ref{th:prop} is a standard fact about
SNS-matrices.
 
\begin{lemma}[{\rm Stability}]
\label{stability}
Given $T>0$, suppose that $\| \epsilon (t) \|_{1,2} \leq h^{q-2}$
for $0 \leq t \leq T$ and $q \geq 6$. 
Then there exists a positive number $B$ that depends on
$T$ and the exact solution $\psi$ only such that for all $0 \leq t \leq T$,
\begin{equation}
\label{Gron}
\frac {d}{dt} \| \epsilon (t) \| _{1,2}  \leq B
   ( h^{q-3/2} + \| \epsilon (t) \|_{1,2})\;.
\end{equation}
The function $B(T)$ can be chosen to be nondecreasing in time.
\end{lemma}
 
 
\begin{theorem} 
\label{th:gibson} 
The maximum number of nonzero entries in a {\rm SNS}-matrix
$S$ of order $n$ equals \[\frac{n^{2}+3n-2}{2}\] with
equality if and only if there exist permutation matrices
such that $P|S|Q=T_{n}$ where
\begin{equation} 
\label{eq:gibson} 
T_{n}=\left[\begin{array}{cccccc} 1&1&\cdots&1&1&1\\
1&1&\cdots&1&1&1\\ 0&1&\cdots&1&1&1\\ 
\vdots&\vdots&\ddots&\vdots&\vdots&\vdots\\ 
0&0&\cdots&1&1&1\\ 0&0&\cdots&0&1&1\end{array}\right]. 
\end{equation} 
\end{theorem} 
 
We note for later use that each submatrix of $T_{n}$ of
order $n-1$ has all 1s on its main diagonal. 
 
We now obtain a bound on the number of nonzero entries of
$S$ in a SNS-matrix pair $(S,C)$ in terms of the degree of
the indicator polynomial. We denote the strictly upper
triangular (0,1)-matrix of order $m$ with all 1s above the
main diagonal by $U_{m}$. The all 1s matrix of size $m$ by
$p$ is denoted by $J_{m,p}$.
 

\begin{definition}
Let $S$ be an isolated invariant set with isolating neighborhood $N$.
An {\em index pair} for $S$ is a pair of compact sets $(N_{1},N_{0})$
with $N_{0} \subset N_{1} \subset N$ such that:
\begin{enumerate}
\item $cl(N_{1} \backslash N_{0})$
is an isolating neighborhood for $S$.
\item $N_{i}$ is positively invariant relative to $N$ for $i=0,1$,
i.e., given
$x \in N_{i}$ and $x \cdot [0,t] \subset N$, then $x \cdot [0,t] \subset
N_{i}$.
\item $N_{0}$ is an exit set for $N_{1}$, i.e. if $x \in N_{1}$,
$x \cdot [0, \infty ) \not\subset N_{1}$, then there is a $T \geq 0$ such
that $x \cdot [0,T] \subset N_{1}$ and $x \cdot T \in N_{0}$.
\end{enumerate}
\end{definition}

\subsection{Numerical experiments} We conducted numerical experiments 
in computing inexact Newton steps for discretizations of a  
{\em modified Bratu problem}, given by  
\begin{eqnarray} 
{\displaystyle \Delta w + c e^w + d{ {\partial w}\over{\partial x} } } 
&=&{\displaystyle f \quad {\rm in}\ D, }\nonumber\\[-1.5ex]
\label{bratu} \\[-1.5ex]
{\displaystyle w }&=&{\displaystyle 0 \quad {\rm on}\ \partial D , } \nonumber
\end{eqnarray} 
where $c$ and $d$ are constants. The actual Bratu problem has $d=0$ and  
$f \equiv0$. It provides a simplified model of nonlinear diffusion  
phenomena, e.g., in combustion and semiconductors, and has been 
considered by Glowinski, Keller, and Rheinhardt \cite{GloKR85}, 
as well as by a number of other investigators; see \cite{GloKR85} 
and the references therein. See also problem 3 by Glowinski and  Keller  
and problem 7 by Mittelmann in the collection of nonlinear model 
problems assembled by Mor\'e \cite{More}. The modified problem  
(\ref{bratu}) has been used as a test problem for inexact Newton 
methods by Brown and Saad \cite{Brown-Saad1}.  

\def\gmres{{GMRES}} 
\def\gmresm{{\rm GMRES($m$)}} 

In our experiments, we took $D = [0,1]\times[0,1]$, $f \equiv0$, 
$c=d=10$, and discretized (\ref{bratu}) using the usual second-order 
centered differences over a $100\times100$ mesh of equally 
spaced points in $D$. In \gmres($m$), we took $m=10$ and used fast  
Poisson right preconditioning as in the experiments in \S2. The computing  
environment was as described in \S2. All computing was done  
in double precision.  
 
\begin{figure}
\includegraphics[width=\columnwidth]{fig}
\caption{Graph of the function $\sin(x)/x$.} 
\label{diff} 
\end{figure}  
%% ak je vas obrazok siroky a chcete ho zobrazit
%% cez dva stlpce, pouzite \begin{figure*}...\end{figure*}
 
In the first set of experiments, we allowed each method to  
run for $40$ {\gmresm} iterations, starting with zero as the initial  
approximate solution, after which the limit of residual norm  
reduction had been reached. The results are shown in Fig.~\ref{diff}.  
In Fig.~\ref{diff}, the top curve was produced by method FD1. 
The second curve from the top is actually a superposition of  
the curves produced by methods EHA2 and FD2; the two curves are 
visually indistinguishable. Similarly, the third curve from  
the top is a superposition of the curves produced by methods EHA4 
and FD4, and the fourth curve from the top, which lies barely above  
the bottom curve, is a superposition of the curves produced by  
methods EHA6 and FD6. The bottom curve was produced by method A. 
 
\begin{table*}
\caption{Statistics over $20$ trials of {\rm GMRES$(m)$} iteration numbers,  
$F$-evaluations, and run times required to reduce the residual norm by  
a factor of $\epsilon$. For each method, the number of {\rm GMRES$(m)$} iterations  
and $F$-evaluations was the same in every trial.} 

\begin{center} \footnotesize
\begin{tabular}{|c|c|c|c|c|c|} \hline  
&& Number of & Number of & Mean Run Time & Standard \\ 
Method & $\epsilon$ & Iterations & $F$-Evaluations& (Seconds) & Deviation \\ \hline 
\lower.3ex\hbox{EHA2} & \lower.3ex\hbox{$10^{-10}$} & \lower.3ex\hbox{26} &  
\lower.3ex\hbox{32} & \lower.3ex\hbox{47.12} & \lower.3ex\hbox{.1048} \\ 
FD2 & $10^{-10}$ & 26 & 58 & 53.79 & .1829 \\ \hline  
\lower.3ex\hbox{EHA4} & \lower.3ex\hbox{$10^{-12}$} & \lower.3ex\hbox{30} &  
\lower.3ex\hbox{42} & \lower.3ex\hbox{56.76} & \lower.3ex\hbox{.1855} \\  
FD4 & $10^{-12}$ & 30 & 132 & 81.35 & .3730 \\ \hline  
\lower.3ex\hbox{EHA6} & \lower.3ex\hbox{$10^{-12}$} & \lower.3ex\hbox{30} &  
\lower.3ex\hbox{48} & \lower.3ex\hbox{58.56} & \lower.3ex\hbox{.1952} \\ 
FD6 & $10^{-12}$ & 30 & 198 & 100.6 & .3278 \\ \hline  
\end{tabular}
\end{center} 
\label{diffstats} 
\end{table*}
%% ak je vasa tabulka uzka a chcete, aby zaberala iba
%% jeden stlpec, pouzite \begin{tabular}...\end{tabular}

In our second set of experiments, we took $c=d=100$ and carried out  
trials analogous to those in the first set above. No preconditioning  
was used in these experiments, both because we wanted to compare 
the methods without preconditioning and because the fast  
Poisson preconditioning used in the first set of experiments is 
not cost effective for these large values of $c$ and $d$. We first  
allowed each method to run for 600 {\gmresm} iterations, 
starting with zero as the initial approximate solution, after which  
the limit of residual norm reduction had been reached. 
 

\section*{Acknowledgments}
The authors thank the anonymous contributors whose work largely
constitutes this sample file. This work has been supported by the 
European Community grant number xxxxx and by the Comenius University
Grant for Young Researchers.

\nocite{*}
\bibliographystyle{apalike}
\bibliography{references}

%% citacie ulozte do suboru references.bib
%% na populaciu zoznamu literatury pouzite program
%%
%% bibtex references
%%
%% po ktorom je potrebne dokument znova zlatexovat

\end{document}
