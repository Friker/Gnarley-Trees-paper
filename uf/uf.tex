\def\uf{Union-Find}
\def\null{\texttt{NULL}}
\def\makeset{$\mathop{\mathbf{makeset}}$}
\def\find{$\mathop{\mathbf{find}}(x)$}
\def\union{$\mathop{\mathbf{union}}(x, y)$}

\section{\uf}
Sú problémy, ktoré vyžadujú spájanie objektov do množín a množín navzájom 
a následné určovanie, do ktorej množiny objekt patrí. Od takejto \emph{
dátovej štruktúry pre disjunktné množiny} očakávame, že si bude udržiavať 
jednoznačného \emph{zástupcu} každej množiny a bude poskytovať 
tieto tri oprácie: 
\begin{itemize}
\item \makeset\ -- vytvorí novú množinu s jedným prvkom, ktorý 
nepatrí do žiadnej inej množiny;
\item \find\ -- nájde zástupcu množiny, v ktorej sa 
prvok $x$ nachádza;
\item \union\ -- vytvorí novú množinu, ktorá obsahuje 
všetky prvky v množinách, ktorých zástupcovia sú $x$ a $y$. Tieto 
množiny zmaže. Ďalej vyberie nového zástupcu novej množiny. Pre 
jednoduchosť, táto operácia predpokladá, že $x$ a $y$ sú 
zástupcovia množín.
\end{itemize}
Vďaka dvom hlavným operáciam \find\ a \union\ 
je táto dátová štruktúra známejšia pod pojmom \emph{\uf}, ktorý 
používame aj my. Medzi najznámejšie problémy, ktoré sa riešia pomocou 
\uf\ patria Kruskalov algoritmus na nájdenie najlacnejšej kostry 
\citep{kruskal} a unifikácia \citep{unif}. Veľmi triviálne použitie je 
zistenie počtu komponentov v grafe.

Vďaka častej asociácií objektov a spájania množín ako vrcholy a hrany grafu 
sa často dátová štruktúra abstraktne reprezentuje ako 
\emph{les} -- množina zakorenených stromov. 
Konkrétnou implementáciou potom býva pole objektov---vrcholov. Ku každému 
objektu sa musí udržiavať smerník $p(x)$ na otca v strome. Smerník zástupcu 
množiny zvyčajne ukazuje na seba ($p(x) = x$). V našej implementácií však 
smerník zástupcu množiny ukazuje na hodnotu \null.

rozne kopresie cesty\\
citovat Walkerov alg.
\\
Yahoo! Walker, Texas ranger, FTW!