\section{Úvod}
Ako ľudia so záujmom o dátové štruktúry sme sa rozhodli pomôcť vybudovať 
dobrý softvér na vizualizáciu algoritmov a dátových štruktúr a obohatiť 
kompiláciu Jakuba Kováča \citep{kuko} o ďalšie dátové štruktúry. 
Vizualizujeme rôznorodé dátové štruktúry. Z binárnych vyvažovaných stromov 
to sú \emph{finger tree} a \emph{reversal tree}, z háld to sú \emph{d-nárna 
halda}, \emph{ľavicová halda}, \emph{skew halda} a \emph{párovacia halda}. 
Taktiež vizualizujeme aj \emph{problém disjuktných množín (union-find 
problém)} a \emph{písmenkový strom (trie)}. 

Okrem vizualizácie prerábame softvér, doplnili sme ho o históriu krokov 
a operácií, jednoduchšie ovládanie a veľa iných vecí, zlepšujúcich 
celkový dojem. Softvér je celý v slovenčine a angličtine a je 
implementovaný v jazyku \texttt{Java}.

\subsection{Vizualizácia}
Dátové štruktúry a algoritmy tvoria základnú, prvotnú časť výučby 
informatiky. Vizualizácia algoritmov a dátových štruktúr je grafické 
znázornenie, ktoré abstrahuje spôsob ako algoritmus a dátové štruktúry 
pracujú od ich vnútornej reprezentácie a umiestnení v pamäti. Je teda 
vyhľadávaná a všeobecne rozšírená pomôcka pri výučbe. Výsledky výskumov 
ohľadne jej efektívnosti sa líšia, od stavu "nezaznamenali sme výrazné 
zlepšenie" po "je viditeľné zlepšenie". \citep{shaffer}

Rozmach vizualizačných algoritmov priniesla najmä Java a jej fungovanie 
bez viazanosti na konkrétny operačný systém. Kvalita vizualizácií sa líši 
a keďže ide o ľahko naprogramovateľné programy, je ich veľa a sú pomerne 
nekvalitné. V takomto množstve je ťažké nájsť kvalitné vizualizácie. 
Zbieraním a analyzovaním kvality sa venuje skupina AlgoViz, ktorá už 
veľa rokov funguje na portále \url{http://algoviz.org/}.

Zaujímavé je pozorovanie, že určovanie si vlastného tempa pri vizualizácií 
je veľká pomôcka. Naopak, ukazovanie pseudokódu alebo nemožnosť určenia si
vlastného tempa (napríklad animácia bez možnosti pozastavenia), takmer 
žiadne zlepšenie neprináša. \citep{shaffer,saraiya}

\subsection*{Motivácia}
Z vyššie uvedeného je jasné, že našou snahou je vytvoriť kvalitnú kompiláciu 
a softvér, ktorý bude nezávislý od operačného systému, bude vyhovovať ako 
pomôcka pri výučbe ako aj pri samoštúdiu a bude voľne prístupný a náležite 
propagovaný. Toto sú hlavné body, ktoré nespĺňa žiaden slovenský a 
len veľmi málo svetových vizualizačných softvérov. Našou hlavnou snahou 
je teda ponúknuť plnohodnotné prostredie pri učení.
%co robime? na co je dobra vizualizacia? (mozete porovnat svet s a bez vizualizacie :))
%co konkretne vizualizujeme? preco to robime? (uz daco take je? [da sa odcitovat shaffer])
%
%Friker, makaj! Zvyšok, pomôž!

