\section{Úvod}
Dátové štruktúry a algoritmy tvoria základnú, prvotnú časť výučby 
informatiky. Vizualizácia algoritmov a dátových štruktúr je grafické 
znázornenie, ktoré abstrahuje od implementačných detajlov a reprezentácie
v pamäti. Je teda vhodnou pomôckou pri výučbe i samoštúdiu. Hoci výsledky
výskumov zatiaľ nedokázali úplne preukázať pedagogickú efektívnosť
vizualizácií \citep{shaffer}, viacero štúdií potvrdilo zvýšený záujem
a zapojenosť študentov \citep{naps02, hundhausen02}.

Ako ľudia so záujmom o dátové štruktúry sme sa rozhodli pomôcť vybudovať 
dobrý softvér na vizualizáciu algoritmov a dátových štruktúr a obohatiť 
kompiláciu Jakuba Kováča \citep{kuko} o ďalšie dátové štruktúry. 
Vizualizujeme rôznorodé dátové štruktúry. Z binárnych vyvažovaných stromov 
to sú \emph{B$^+$-strom}, \emph{strom s prstom} a \emph{strom s reverzami}, z háld to sú \emph{d-árna 
halda}, \emph{ľavicová halda}, \emph{skew halda} a \emph{párovacia halda}. 
Taktiež vizualizujeme aj \emph{union-find problém} a 
\emph{písmenkový strom (trie)}. 

Okrem vizualizácie softvér prerábame a neustále vylepšujeme.
Doplnili sme ho o históriu krokov a operácií, jednoduchšie ovládanie
a veľa ďalších funkcií. Softvér je celý v slovenčine aj angličtine a je 
implementovaný v jazyku \texttt{Java}. Dostupný je na stránke
\hbox{\url{http://people.ksp.sk/~kuko/gnarley-trees}} vo forme appletov
s jednotlivými dátovými štruktúrami, a tiež vo forme samostatného programu,
ktorý obsahuje všetky dátové štruktúry a je určený na používanie offline.
%, kde má každá dátová štruktúra 

Rozmach vizualizačných algoritmov priniesla najmä Java a jej fungovanie 
bez viazanosti na konkrétny operačný systém. Kvalita vizualizácií sa líši 
a keďže ide o ľahko naprogramovateľné programy, je ich veľa a sú pomerne 
nekvalitné. V takomto množstve je ťažké nájsť kvalitné vizualizácie. 
Zbieraním a analyzovaním kvality sa venuje skupina AlgoViz, ktorá už 
veľa rokov funguje na portále \url{http://algoviz.org/}.

Zaujímavé je pozorovanie, že určovanie si vlastného tempa pri vizualizácií 
je veľká pomôcka. Naopak, ukazovanie pseudokódu alebo nemožnosť určenia si
vlastného tempa (napríklad animácia bez možnosti pozastavenia), takmer 
žiadne zlepšenie neprináša \citep{shaffer,saraiya}.

Našou snahou je vytvoriť kvalitný softvér nezávislý od operačného systému, 
ktorý bude vyhovovať ako pomôcka pri výučbe ako aj pri samoštúdiu a bude
voľne prístupný. Toto sú hlavné body, ktoré nespĺňa žiaden slovenský a 
len veľmi málo svetových vizualizačných softvérov. Našou hlavnou snahou 
je teda ponúknuť plnohodnotné prostredie pri učení.