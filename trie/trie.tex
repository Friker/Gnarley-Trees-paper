
\def\k{k}
%\def\kluc{kľúč}
\def\put{$\mathop{insert}(\k)$}
\def\find{$\mathop{find}(\k)$}
\def\delete{$\mathop{delete}(\k)$}
\def\trie{trie}

\section{Písmenkový strom}
\emph{Abeceda} je množina \emph{znakov}. \emph{Slovo} je postupnosť znakov 
z danej abecedy. \emph{Písmenkový strom} je \emph{zakorenený strom}, v ktorom 
sú vrcholy prázdne a každá hrana obsahuje práve jeden znak z abecedy alebo 
\emph{ukončovací znak}. Strom 
obsahuje slová nazývané \emph{kľúče} a to tak, že kľúč je rozdelený na znaky 
abecedy a tie sú umiestnené na jednotlivých hranách tak, že po prejdení cesty 
z koreňa do vrchola prečítame daný kľúč. 

Oproti binárnym vyhľadávacím stromom je hlavný rozdiel v tom, že \emph{kľúče} 
nie sú uložené vo vrcholoch, ale samotná poloha v strome určuje kľúč. To, či 
je kľúč v strome uložený určuje \emph{ukončovací znak}. Môže to byť ľubovoľný 
symbol, ktorý sa nenachádza v abecede. My budeme používať ako abecedu veľké 
znaky anglickej abecedy a ukončovací znak značiť dolárom. ($\$$)

\paragraph{Popis}
Vzhľadom na svoj vznik sa písmenkový strom využíva ako \emph{asociatívne pole 
(slovník)} a teda poskytuje tieto tri operácie:
\begin{itemize}
\item $\mathop{\mathbf{insert}}(\k)$ -- pridá do stromu kľúč $\k$;
\item $\mathop{\mathbf{find}}(\k)$ -- zistí, či sa v strom kľúč $\k$ nachádza;
\item $\mathop{\mathbf{delete}}(\k)$ -- odstráni zo stromu kľúč $\k$ a 
prípadne vyrieši zmeny v strome.
\end{itemize}
Operácia \put\ začína v koreni. Zoberie kľúč $k$ a spraví z neho postupnosť 
znakov. Na koniec postupnosti zaradí ukončovací znak. 
Následne začne vykonávať cyklus, v ktorom vyberie z postupnosti ten znak, 
ktorý je na rade a presunie sa z momentálneho vrchola po príslušnej hrane. 
Ak hrana neexistuje pridá sa do stromu nový vrchol a príslušná hrana. Cyklus 
sknčí, keď sa vyčerpajú všetky znaky a teda sa do stromu vloží celý kľúč a 
ukončovací znak.

\begin{figure*}
\includegraphics[width=2\columnwidth]{obrazky/trieinsert.png}
\caption{\emph{Operácia \put.} Vo vizualizácií naznačujeme, kade sa 
presunieme.} 
\label{img:trieinsert} 
\end{figure*}

Operácia \find\ začína hľadať kľúč $k$ taktiež v koreni. Pridá ku 
postupnosti spravenej z kľúča ukončovací znak (podobne ako pri \put) a 
spustí sa z koreňa podľa postupnosti znakov. Ak hrana, po ktorej sa máme
spustiť neexistuje, daný kľúč sa v strome nenachádza. Ak prečítame celú 
postupnosť, daný kľúč sa v strome nachádza.

Operácia \delete\ najprv nájde hľadaný kľúč pomocou operácie \find\ a potom 
už len stačí odstrániť hranu s ukončovacím symbolom a vrchol, ktorý bol 
na nej zavesený. V tomto štádiu sa nám môže stať, že v strome ostane 
takzvaná \emph{mŕtva vetva} -- nie je ukončená ukončovacím znakom. 
Keďže si práve pamätáme smerník na mrtvého syna môžeme ho odtrhnúť. A takto 
pokračovať vyššie až kým neutrhneme celú vetvu.

\begin{figure*}
\includegraphics[width=2\columnwidth]{obrazky/triedeletebig.png}
\caption{\emph{Operácia \delete.} Najprv zistíme, či je kľúč v strome, 
potom ho zmažeme a potom zmažeme mŕtvu vetvu, ktorá je vyznačená 
červenou farbou.} 
\label{img:triedelete} 
\end{figure*}

\paragraph{Použitie}
Písmenkový strom sa používa v rôznych modifikáciach a na rôzne účely. O 
zavedenie pojmov sa zaslúžil \citet{fredkin}, ktorý používal operácie 
$\mathop{storage}$ (\put), $\mathop{retrieval}$ (\find) 
a $\mathop{deletion}$ (\delete) a dátovú štruktúru nazýval \emph{trie 
memory}, keďže išlo naozaj o spôsob uloženia dát v pamäti. Výraz \emph{trie}%
\footnote{Z anglického re\emph{trie}val -- získať.} 
sa ujal a tak budeme písmenkový strom označovať aj v ďalšom texte.
Vďaka tomu, že písmenkový strom udržiava spoločné \emph{prefixy} kľúčov sa 
nazýva aj \emph{prefixový strom}.

O niečo neskôr \citet{knuth} uviedol vo svojej knihe ako príklad na \trie\ 
vreckový slovník. V tom istom diele uviedol aj možnosť komprimovania vetiev 
a možnosť prerobenia $m$-árneho \trie\ na binárny. 
Pre silnú podobnosť s \emph{konečnými automatmi} vznikli prerobenie a 
implementácie \trie\ pomocou komprimovaných automatov. Tieto majú použitie 
najmä ako pomôcky pri kontrole pravopisu \citep{ca}. Každý jazyk obsahuje 
veľa predpôn a prípon a ich kombinácií, preto takéto automaty zaberajú 
málo miesta. 

Priamočiare je použitie \trie\ na utriedenie poľa slov. Všetky slová sa 
pridajú do stromu a potom sa spraví \emph{preorderový prechod} \trie. 
Túto myšlienku spracoval \citet{burstsort1} a veľmi výrazne zrýchlil triedenie 
dlhých zoznamov slov. Neskôr tento algoritmus vylepšil \citet{burstsort2}.
Špeciálnym použitím \trie\ je zmena jeho prefixovej charakteristiky 
na suffixovú a vytvorenie tzv. \emph{suffixového stromu}. Táto štruktúra má 
veľmi veľa praktických využití \citep{gusfield}. 

%Friker, neobzeraj baby a pracuj! 
%Nejaké citovateľné práce o trie?

