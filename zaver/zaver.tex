\section{Záver}
%work in progress; co sme spravili - uz je v uvode?, preco sme lepsi - sme vobec lepsi (ako galles)?, 
%co este chceme/treba spravit - maybe DONE, co je rozrobene - v podstate to, co chceme spravit - same as previous?
%\\

V súčasnosti je málo programov, ktoré by prinášali komplexnejší prehľad 
využívaných dátových štruktúr. Na niektoré dokonca vizualizácia doposiaľ 
neexistuje. Je veľa appletov, ktoré implementujú niektoré algoritmy, avšak 
ich nedostatkom býva neprehľadnosť vizualizácie a hlavne okrem základných 
operácií dátovej štruktúry takmer žiadna interaktivita. 

Do budúcna sa plánujeme zaoberať vizualizáciou ďalších dátových 
štruktúr a program obohatíme aj o známe algoritmy. Plánujeme implementovať 
\emph{linking-cutting stromy}, \emph{intervalové stromy}, \emph{soft haldu} 
a niektoré \emph{perzistentné dátové štruktúry}.

Budeme pokračovať v dopĺňaní histórie krokov do všetkých dátových štruktúr, 
ale aj vo vylepšení používateľského rozhrania, refaktorovaní zdrojového kódu 
a inými softvérovými vylepšeniami. Našim cieľom je čo najviac zjednodušiť 
prácu s programom, spraviť ho čo najviac používateľsky prístupným a 
zrozumiteľným, a tak zefektívniť výučbu jednotlivých dátových štruktúr, 
resp.\ spraviť ju zábavnejšou.
